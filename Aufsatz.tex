%!TeX program = xelatex
%This is a sample LaTeX input file.  (Version of 11 April 1994.)
%
% A '%' character causes TeX to ignore all remaining text on the line,
% and is used for comments like this one.

\documentclass[12pt]{article}

\title{\textbf{分析導論 學期報告}}  % Declares the document's title.
\author{電機三~徐悅聲(B02901084)\\指導:張志中~教授}      % Declares the author's name.
\date{\today}      % Deleting this command produces today's date.

\usepackage[a4paper, margin=2 cm]{geometry}
\usepackage{fontspec, sectsty, titling, amsmath, booktabs, diagbox} %加這個就可以設定字體 
\usepackage{xeCJK} %讓中英文字體分開設置
\setCJKmainfont[BoldFont=Kaiti TC Bold]{Kaiti TC}

\defaultfontfeatures{Ligatures=TeX}

%\setCJKmainfont{HanWangGSolid06cut1} %設定中文為系統上的字型,而英文不去更動,使用原TeX字型
 % TTF 字型檔名: wtg-06cut1.ttf 超黑俏皮動物  PostScript 名: HanWangGSolid06cut1        CJK/FOP 字型名:hwgsc1

\XeTeXlinebreaklocale "zh" %這兩行一定要加,中文才能自動換行
\XeTeXlinebreakskip = 0pt plus 1pt %這兩行一定要加,中文才能自動換行

%\allsectionsfont{\sffamily}
\setlength{\droptitle}{-5em}

\begin{document}             % End of preamble and beginning of text.
\maketitle                   % Produces the title.

今天我想談談的是我會選擇雙主修數學系的原因。起初只是與好友一席玩笑話。

當初在大一升大二暑假之前,大家興致勃勃地討論着下學期計劃要修什麼樣的課,我最好的朋友當時手上有兩本書,一是Friedberg/Insel/Spence三人所著的Linear Algebra,二是鼎鼎大名的Rudin。兩本都是數學系大二必修的用書。我特別覺得它們精美,因爲我本科的工程用書往往笨重如磚,設計缺乏美感,內容不外乎例題導向,沒別的,就是與高中換湯不換藥的解題演練而已,非常之枯燥。而這兩本書卻給我一種很欠讀的感受,擺在眼前簡直實是在引誘你翻開它。它們篇幅精簡適中,結構嚴明,再加上 \LaTeX 的排版又出奇地適合閱讀。

我朋友後來因爲自己的規劃,跑去修哲學的課去了。他看我好像很喜歡那兩本書,就隨口一句:“不如你去申請看看數學系雙主修,成功了兩本書就交給你。”殊不知在數學系開放的態度下要取得雙主修資格並不是那麼困難,但他還是很服氣地把兩本書交到了我手上。於是我從此開啓了“不務正業”的道路。我進入數學系的第一門課就是志中老師的分析導論,一接觸我從此改變對系統知識(至少在數學方面)的認識,


\end{document}               % End of document.
