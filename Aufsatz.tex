%!TeX program = xelatex
%This is a sample LaTeX input file.  (Version of 11 April 1994.)
%
% A '%' character causes TeX to ignore all remaining text on the line,
% and is used for comments like this one.

\documentclass[12pt]{article}

\title{\textbf{分析導論 學期報告}}  % Declares the document's title.
\author{電機三~徐悅聲(B02901084)\\指導:張志中~教授}      % Declares the author's name.
\date{\today}      % Deleting this command produces today's date.

\usepackage[a4paper, margin=2 cm]{geometry}
\usepackage{fontspec, sectsty, titling, amsmath, booktabs, diagbox} %加這個就可以設定字體 
\usepackage{xeCJK} %讓中英文字體分開設置
\setCJKmainfont[BoldFont=Kaiti TC Bold]{Kaiti TC}

\defaultfontfeatures{Ligatures=TeX}

%\setCJKmainfont{HanWangGSolid06cut1} %設定中文為系統上的字型,而英文不去更動,使用原TeX字型
 % TTF 字型檔名: wtg-06cut1.ttf 超黑俏皮動物  PostScript 名: HanWangGSolid06cut1        CJK/FOP 字型名:hwgsc1

\XeTeXlinebreaklocale "zh" %這兩行一定要加,中文才能自動換行
\XeTeXlinebreakskip = 0pt plus 1pt %這兩行一定要加,中文才能自動換行

%\allsectionsfont{\sffamily}
\setlength{\droptitle}{-5em}

\begin{document}             % End of preamble and beginning of text.
\maketitle                   % Produces the title.

今天我想談談的是我會選擇雙主修數學系的原因。起初只是與好友一席玩笑話。\\

當初在大一升大二暑假之前,大家興致勃勃地討論着下學期計劃要修什麼樣的課,我最好的朋友當時手上有兩本書,一是Friedberg/Insel/Spence三人所著的Linear Algebra,二是鼎鼎大名的Rudin。兩本都是數學系大二必修的用書。我特別覺得它們精美,因爲我本科的工程用書往往笨重如磚,設計缺乏美感,內容不外乎例題導向,沒別的,就是與高中換湯不換藥的解題演練而已,非常之枯燥。而這兩本書卻給我一種很欠讀的感受,擺在眼前簡直實是在引誘你翻開它。它們篇幅精簡適中,結構嚴明,再加上 \LaTeX 的排版又出奇地適合閱讀。\\

我朋友後來因爲自己的規劃,跑去修哲學的課去了。他看我好像很喜歡那兩本書,就隨口一句:“不如你去申請看看數學系雙主修,成功了兩本書就送給你。”玩笑歸玩笑,最後他還是很服氣地把兩本書交到了我手上,於是我從此開啓了“不務正業”的道路。​我進入數學系的第一門課就是志中老師的分析導論,一接觸我從此改變對系統知識(至少在數學方面)的認知。此後每每在順著定理推論或寫習題的時候,我總是會暗地裡驚呼:我是真的有在思考,而不只是動手放空地計算而已耶!也是那時起,我才第一次真正感受到自己是在大學裡學習,而不是個四不像的職業訓練所。\\

一直到現在大三了,在數學系的確是受到了不少學業上的挫折,但我很滿意這個決定以及現在投入的生活。要說我都學了些什麼或數學給了我什麼益處,或更直接一點問:數學有什麼用?這個問題讓我聯想起這學期修代數時陳其誠老師說的一席話:“數學本來就沒什麼用了,我們不需要假裝她很有用。”我認為這句話十分有道理,而覺得更可以這樣詮釋:“數學並不需要透過有用處這件事來展現她的價值。”就如同不會有人說藝術實務上很有用一樣,數學的成果也許要花上不知多久的時間才會在別的知識領域出現應用,但這門學問的價值是超乎這個層次的。前一段時間讀了黃武雄教授在一篇”自然中的理性“文章中所寫道:“本質探討也容易使人從傳統價值或現實權威中解放出來。 透過直接觀察現象,建立事實,而進入理則世界的本質,可以使人超越世俗的價值,自理性的信仰中去否定現世的權威。......一個意氣風發的少年,隨著數學的引導,進入理則世界的本質探討,由於理性的嚴謹與純淨,他會肯定獨立思考的內在價值,從而由根本質疑世俗的傳統規範與權威。”這句話直接又切中要點地概括了我在數學系的體驗與收穫,透過公設化、定理化的思考訓練,將自己沈靜入純粹的理則世界,這絕對是一所大學能帶給學子最重要的禮物與精神。\\

思考是一輩子的事。我理想的人生圖像就希望是由書本與富足的精神生活構成(也當然需要足夠的物質支撐),也因此我在認真思考以數學作為志業這件事。儘管我不認為自己微小的存在能帶來任何貢獻,但能倘佯在純粹的理則世界裡並給自己尋著一個位置,仍然會是我永遠努力的目標與夢想。總之,我很感謝送我書的朋友。


\end{document}               % End of document.
